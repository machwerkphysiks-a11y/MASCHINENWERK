\section{Delta Zero Marker}

This application illustrates the use of $\Delta_0$ as a formal
decision marker in computations that cross a projection boundary.

\subsection{Setup}

Let
\[
\Pi : \mathcal{U} \rightarrow m_2
\]
be a projection into an observable calculation domain.

Consider a formally well-defined expression
\[
E(u)
\]
with
\[
u \in \mathcal{U}
\]

\subsection{Computation}

Assume the computation of $E(u)$ proceeds without algebraic failure.
All intermediate steps remain formally valid.

However, beyond a certain configuration,
the result cannot be uniquely associated with an observable state in $m_2$.

\subsection{Delta Zero Assignment}

At this point, the computation is marked by
\[
E(u) \;\Rightarrow\; \Delta_0
\]

This does not terminate the calculation.
It classifies the result as non-interpretable in physical terms.

\subsection{Interpretation Rule}

$\Delta_0$ is not a value, limit, or approximation.

It indicates that the computation has crossed the boundary
where physical admissibility ends.

\subsection{Result}

All formal manipulations remain valid.

Only physical interpretation is suspended.
