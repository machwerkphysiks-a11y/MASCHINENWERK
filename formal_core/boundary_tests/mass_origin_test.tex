\section{Mass Origin Test}

This test examines whether mass-like quantities
can be treated as intrinsic properties
or must be regarded as projection-dependent relational quantities.

\subsection{Setup}

Let
\[
\Pi : \mathcal{U} \rightarrow m_2
\]
be an injective projection within a stable domain.

Assume a relational configuration $u \in \mathcal{U}$
associated with a mass parameter $m(u)$.

\subsection{Projection Dependence}

Mass is observable only through relational effects
within $m_2$.

If distinct relational configurations
\[
u_1 \neq u_2
\]
satisfy
\[
\Pi(u_1) = \Pi(u_2),
\]
then any intrinsic mass assignment must satisfy
\[
m(u_1) = m(u_2)
\]
to remain admissible.

\subsection{Boundary Behavior}

At or beyond the black boundary $\Sigma$,
multiple relational configurations
collapse into identical observable states.

Mass values that distinguish such configurations
are not projection-invariant.

\subsection{Result}

Mass is admissible only as a relationally stabilized quantity
within injective projection domains.

Intrinsic mass definitions beyond $m_2$
are formally definable but physically inadmissible.

\subsection{Conclusion}

This test enforces mass as a derived relational quantity,
not as an ontological primitive.
