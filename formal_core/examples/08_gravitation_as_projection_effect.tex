\section{Gravitation as a Projection Effect}

This application illustrates
how gravitation can be treated
as a consequence of projection structure
rather than a fundamental force.

\subsection{Setup}

Let
\[
\Pi : \mathcal{U} \rightarrow m_2
\]
be a projection into an observable calculation space.

Assume a relational configuration
\[
u \in \mathcal{U}
\]
with stable internal rate relations.

\subsection{Relational Density}

In $\mathcal{U}$,
relations possess no spatial extension.
There is no notion of distance or curvature.

What is observed as mass density in $m_2$
corresponds to relational concentration
under projection.

\subsection{Projection Asymmetry}

Projection compresses relational structure
into spatial representations.

Regions of high relational connectivity
appear as massive objects
in $m_2$.

\subsection{Attraction Without Force}

Observable attraction arises
because projected trajectories
converge toward regions
where projection overlap increases.

No force is transmitted in $\mathcal{U}$.
No interaction propagates between objects.

\subsection{Geometric Interpretation}

Curvature in $m_2$
is a geometric encoding
of projection asymmetry.

The apparent gravitational field
reflects loss of relational degrees of freedom.

\subsection{Black Boundary Context}

Near projection limits,
injectivity weakens.

Gravitational anomalies emerge
as boundary artifacts,
not as violations of relational consistency.

\subsection{Result}

Gravitation is interpreted
as a stable projection artifact
of relational structure,
not as a fundamental interaction.
