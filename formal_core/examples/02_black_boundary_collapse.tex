\section{Black Boundary Collapse}

This application illustrates how loss of injectivity
produces a black boundary without introducing singularities
or physical discontinuities.

\subsection{Setup}

Let
\[
\Pi : \mathcal{U} \rightarrow m_2
\]
be a projection from the relational reference space
into an observable calculation domain.

Assume a family of relational states
\[
\{u_i\} \subset \mathcal{U}.
\]

\subsection{Injectivity Loss}

Suppose there exist at least two distinct states
\[
u_1 \neq u_2
\]
such that
\[
\Pi(u_1) = \Pi(u_2).
\]

The observable representation is identical,
although the underlying relational configurations differ.

\subsection{Definition of the Boundary}

The set of all relational states satisfying this condition
defines the black boundary:
\[
\Sigma := \{ u \in \mathcal{U} \mid \Pi \text{ is non-injective at } u \}.
\]

\subsection{Structural Consequences}

Within the black boundary:

\begin{itemize}
\item observable quantities remain well-defined,
\item formal calculations remain consistent,
\item relational reconstruction becomes impossible.
\end{itemize}

No contradiction arises.
Only uniqueness is lost.

\subsection{Physical Meaning}

The black boundary is not a physical object,
not a horizon, and not a singularity.

It is a structural property of the projection.

What collapses is not reality,
but the ability to assign a unique physical interpretation.

\subsection{Result}

The black boundary marks the exact point where
physical description transitions into formal continuation,
without requiring any modification of the underlying mathematics.
