\section{Three-Reaction Dynamics as Comparison Tool}

This application illustrates how Three-Reaction Dynamics (TRD)
provides a minimal and stable comparison structure
for physical statements.

\subsection{Motivation}

Pairwise comparisons of processes are structurally ambiguous.
They cannot distinguish between intrinsic change
and contextual rescaling.

TRD introduces a third reference
to stabilize relational assessment.

\subsection{Setup}

Let three distinguishable processes
have rates
\[
R_1,\; R_2,\; R_3.
\]

Only ratios are physically admissible:
\[
X_{ij} = \frac{R_i}{R_j}.
\]

\subsection{Closed Relational Structure}

With three processes,
the set of ratios
\[
\{X_{12}, X_{23}, X_{31}\}
\]
forms a closed comparison loop.

This loop remains invariant under
global rescaling
\[
R_i \mapsto \lambda R_i.
\]

\subsection{Stability Criterion}

A relational statement is stable if
all derived ratios remain consistent
under reference exchange.

TRD allows detection of:

\begin{itemize}
\item context shifts,
\item hidden rescalings,
\item incompatible model assumptions.
\end{itemize}

\subsection{Beyond Pairwise Logic}

TRD does not describe dynamics in time.
It describes consistency of comparison.

It separates:

\begin{itemize}
\item process variation,
\item reference variation,
\item projection effects.
\end{itemize}

\subsection{Relation to Boundary Regions}

Near black boundaries,
pairwise relations lose interpretability.

TRD remains applicable
because it does not require
unique inverse reconstruction.

\subsection{Result}

TRD is the minimal formal mechanism
for comparing physical statements
without introducing absolute quantities
or ontological commitments.
