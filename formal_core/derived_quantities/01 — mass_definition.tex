\section{Mass}

Mass is not treated as an intrinsic property of an object.

Within the MACHWERK framework, mass is a
\textbf{stable projection value}
of relational process configurations.

\subsection{Formal Character}

Mass arises from the comparison of process rates
within an injective projection domain.

It is defined only relative to:
\begin{itemize}
\item a relational reference structure
\item a projection $\Pi$
\item a validity domain $m_2$
\end{itemize}

\subsection{Admissibility}

A mass statement $M$ is physically admissible if and only if:
\[
M \in m_2
\]
and remains invariant under admissible reparameterizations
of the underlying relations.

\subsection{Boundary Behavior}

At a Schwarzgrenze, multiple relational configurations
may project to identical mass values.

Beyond this point:
\begin{itemize}
\item mass remains formally writable
\item mass loses physical interpretability
\end{itemize}

\subsection{Non-Assumptions}

Mass is not assumed to be:
\begin{itemize}
\item fundamental
\item conserved by necessity
\item independent of projection
\end{itemize}
