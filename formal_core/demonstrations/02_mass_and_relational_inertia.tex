\section{Mass and Relational Inertia}

\subsection{Problem Class}

Mass is traditionally treated as an intrinsic property of objects.
Within the MACHWERK framework, mass is analyzed as a
\emph{relationally inferred quantity} arising from projected process behavior.

The question is not what mass \emph{is},
but under which formal conditions mass statements are admissible.

\subsection{Relational Background}

In the relational full space $\mathcal{U}$,
no element carries an intrinsic mass parameter.

There exist only relations between processes,
expressed through interaction rates, resistance to change,
and comparative responses to external coupling.

Mass does not exist in $\mathcal{U}$ as a primitive.

\subsection{Projection into an Observable Domain}

Let
\[
\Pi_m : \mathcal{U} \rightarrow m_2
\]
be a projection into an observable calculation domain
where mass-like quantities are defined operationally.

Operationally, mass is inferred from ratios such as:
\[
m \sim \frac{F}{a}
\]
where force and acceleration themselves are projected quantities.

\subsection{Relational Ambiguity}

Distinct relational configurations
\[
u_1 \neq u_2 \in \mathcal{U}
\]
may yield identical projected mass values:
\[
\Pi_m(u_1) = \Pi_m(u_2)
\]

This occurs whenever different relational coupling structures
produce the same resistance-to-change behavior
under the chosen projection.

Thus, $\Pi_m$ is not globally injective.

\subsection{Black Boundary for Mass}

Define
\[
\Sigma_m := \{ u \in \mathcal{U} \mid \Pi_m \text{ is non-injective} \}
\]

Within $\Sigma_m$, mass remains a valid computational quantity,
but it no longer uniquely identifies a relational configuration.

Mass becomes an equivalence class label, not a state descriptor.

\subsection{CRA Admissibility}

A mass statement $S(m)$ is physically admissible
if and only if it depends exclusively on projected observables
and remains invariant across all relational states
mapped to the same mass value.

Any statement attributing additional structure
to mass beyond its projected definition
violates the CRA axiom.

\subsection{Formal Conclusion}

Mass in MACHWERK is not an intrinsic substance.

It is a projection-stable observable
whose physical meaning is restricted
to domains where relational ambiguity does not matter.

Beyond the black boundary,
mass remains calculable
but loses ontological specificity.
