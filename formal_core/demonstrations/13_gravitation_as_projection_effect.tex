\section{Gravitation as a Projection Effect}

\subsection{Problem Class}

Gravitation is usually modeled
as a force or curvature.

MACHWERK treats gravitation
as a projection artifact.

\subsection{Relational Origin}

In $\mathcal{U}$,
only comparative relational densities exist.

There is no force
and no geometric background.

\subsection{Emergence}

Let
\[
\Pi_g : \mathcal{U} \rightarrow m_2
\]

Regions of high relational redundancy
project as attractive structures.

Gravitation measures
relational compression under projection.

\subsection{No Force Interpretation}

Gravitation does not act.
It reflects how relations cluster
under injective projection.

\subsection{Black Boundary}

Near $\Sigma_g$,
distinct relational structures
project identically.

This produces singular behavior
without ontological singularities.

\subsection{Conclusion}

Gravitation is not fundamental.
It is a structural consequence
of relational projection.
