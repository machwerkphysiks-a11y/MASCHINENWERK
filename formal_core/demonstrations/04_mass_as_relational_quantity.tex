\section{Mass as a Relational Quantity}

\subsection{Problem Class}

Mass is traditionally treated as an intrinsic property of objects.
Within the MACHWERK framework, mass is treated as a
\emph{relationally projected quantity},
not as an inherent attribute.

The question is not what mass is,
but under which conditions mass statements are admissible.

\subsection{Relational Background}

In the relational full space $\mathcal{U}$,
there are no objects with intrinsic properties.

There exist only relations between processes,
expressed through interaction structure and comparison.

Mass does not exist in $\mathcal{U}$.

\subsection{Projection into Observable Domains}

Let
\[
\Pi_m : \mathcal{U} \rightarrow m_2
\]
be a projection into an observable calculation domain
where inertial and gravitational parameters are defined.

Operationally, mass emerges from resistance relations:
\[
m \sim \frac{F}{a}
\]
where both force $F$ and acceleration $a$
are themselves projected quantities.

\subsection{Relational Interpretation}

Different relational configurations
may produce identical projected mass values.

Thus, for
\[
u_1 \neq u_2 \in \mathcal{U}
\]
it may hold that
\[
\Pi_m(u_1) = \Pi_m(u_2)
\]

Projected mass does not uniquely identify
the underlying relational structure.

\subsection{Black Boundary for Mass}

Define
\[
\Sigma_m := \{ u \in \mathcal{U} \mid \Pi_m \text{ is non-injective} \}
\]

Within $\Sigma_m$, mass remains a usable calculation parameter,
but loses reconstructive uniqueness.

Mass values beyond this boundary
cannot be interpreted as intrinsic properties.

\subsection{CRA Admissibility}

A statement $S(m)$ is physically admissible
only if it depends exclusively on projected observables
and remains invariant under changes of relational configuration
that preserve $\Pi_m$.

Any attribution of absolute mass
violates the CRA axiom.

\subsection{Formal Conclusion}

Mass in MACHWERK is not a substance.

It is a projection-stable ratio
arising from relational interaction structure.

Mass is meaningful where projections are injective,
and formally ambiguous beyond the black boundary.
