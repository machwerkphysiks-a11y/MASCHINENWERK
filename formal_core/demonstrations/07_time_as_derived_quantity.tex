\section{Time as a Derived Quantity}

\subsection{Problem Class}

Time is commonly treated as a fundamental background parameter.
Within the MACHWERK framework, time is not fundamental.

Time is a derived relational quantity.

The framework does not eliminate time,
but constrains where time-based statements are admissible.

\subsection{Absence of Time in $\mathcal{U}$}

The relational full space $\mathcal{U}$ contains:

\begin{itemize}
\item no global time parameter
\item no absolute ordering of events
\item no privileged temporal metric
\end{itemize}

Only relations between processes exist.
There is change, but no clock.

\subsection{Emergence via Projection}

Let
\[
\Pi_t : \mathcal{U} \rightarrow m_2
\]
be a projection into an observable calculation space
where repeatable relational comparisons are possible.

Time emerges as a ratio of process rates:
\[
t \sim \frac{R_{\text{reference}}}{R_{\text{process}}}
\]

A clock is therefore not a carrier of time,
but a stabilized comparison process.

\subsection{Relational Interpretation}

Time values encode ordered relational change,
not universal flow.

Different relational configurations may project
to identical time readings:
\[
\Pi_t(u_1) = \Pi_t(u_2)
\quad \text{with} \quad u_1 \neq u_2
\]

Observed simultaneity does not imply relational identity.

\subsection{Local Validity}

Time ordering is local to a projection domain.
There is no requirement that different projections
share a common temporal structure.

This directly explains:

\begin{itemize}
\item observer-dependent time dilation
\item synchronization breakdowns
\item non-global temporal ordering
\end{itemize}

without introducing contradictions.

\subsection{Black Boundary for Time}

Define
\[
\Sigma_t := \{ u \in \mathcal{U} \mid \Pi_t \text{ is non-injective} \}
\]

Beyond $\Sigma_t$:

\begin{itemize}
\item process relations persist
\item temporal labels remain computable
\item unique causal ordering is lost
\end{itemize}

Time continues numerically,
but loses interpretive uniqueness.

\subsection{CRA Constraint}

A temporal statement $S(t)$ is admissible
only if it depends solely on projected relational ratios
and is invariant under relational substitutions
that preserve $\Pi_t$.

Claims of absolute or universal time
are physically inadmissible under CRA.

\subsection{Formal Conclusion}

Time in MACHWERK is not a background variable.

It is a projection-dependent relational parameter
emerging from stabilized rate comparisons.

Beyond the black boundary,
time remains calculable but ceases to be uniquely meaningful.
